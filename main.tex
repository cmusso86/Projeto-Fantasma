\documentclass[a4paper, 12pt]{article} %tipo de documento e opções gerais
\usepackage[utf8]{inputenc} %codificação
\usepackage[T1]{fontenc} %codificação
\usepackage[portuguese]{babel}  %idioma
\usepackage{amsmath,amssymb,amsfonts,amsthm} %padrões em formato matemático
\usepackage[a4paper,top=3cm,bottom=2cm,left=3cm, right=2cm]{geometry} %margens
\usepackage{indentfirst} %primeiro parágrafo com margem
\usepackage{float} %fixar imagens e tabelas
\usepackage{multicol} %várias colunas
\usepackage{multirow} %várias linhas
\usepackage{graphicx} %colocar imagens 
\usepackage{anyfontsize} %qualquer tamanho de letra
\usepackage{setspace} %espaçamento
\usepackage[titles]{tocloft} %padrões de sumário
\usepackage{fontspec} %outros tipos de fonte
\usepackage{fancyhdr} %padronizar o formato do Header
\usepackage[resetlabels,labeled]{multibib} %bibliografia
\usepackage{newfloat} %alterar nome do quadro
\usepackage{tabls} %pacote para espaçamento de linhas
%\usepackage{titlesec}
\usepackage{makecell}
\usepackage{shadowtext}
\usepackage{eso-pic,graphicx}
\usepackage{tikz} %trabalhar com imagens e add background
\usepackage[absolute,overlay]{textpos}
\usepackage{calc} %para funçoes como \widthof
\usepackage{hyperref}
\usepackage{afterpage} % contracapa

\newcommand\blankpage{% código para a contracapa
    \null
    \thispagestyle{empty}%
    \addtocounter{page}{-1}%
    \newpage}

%definindo novo estilo de pagina
\makeatletter
\def\ps@Padrao{
    %número da página com cor clara e no canto
    \def\@oddfoot{\textcolor{white}{\null\hfill\thepage}}%
    \def\@evenfoot{\thepage}%
    %definindo cabeçado para o canto
    \def\@evenhead{\null\hfil\slshape\leftmark }%
    \def\@oddhead{{\slshape\rightmark\hfill \includegraphics[scale=0.2]{estat.png}}}} %cabeçalho
\makeatother

\pagestyle{Padrao}

\setmainfont{Arial} %fonte arial
\setstretch{1.5} %espaçamento
\setlength\tablinesep{5pt} %espaço entre as células da tabela

% ALTERANDO O TÍTULO DAS TABELAS E FIGURAS
\addto\captionsenglish{%
  \renewcommand\tablename{Tabela}
  \renewcommand\figurename{Figura}
}
\DeclareFloatingEnvironment[listname=loq, listname={Lista de Quadros}]{quadro}

% ALTERANDO O SUMÁRIO
\makeatletter
\renewcommand\tableofcontents{
  \null\hfill\textbf{\Large\contentsname}\hfill\null\par
  \@mkboth{\MakeUppercase\contentsname}{\MakeUppercase\contentsname}%
  \@starttoc{toc}}
\makeatother	
\addto\captionsenglish{
  \renewcommand{\contentsname}{Sumário}
  }

\clearpage

\begin{document}
\begin{titlepage}

\center
\tikz[remember picture,overlay] \node[opacity=1,inner sep=0pt] at (current page.center){\includegraphics[width=\paperwidth,height=\paperheight]{capa.png}};

\begin{minipage}{16cm}
\begin{flushright}
% posições do título (substituir o segundo parâmetro do begin{textblock}
% 1 linha:  (4cm, 8.38cm)   nao esquecer de cuidar
% 2 linhas: (4cm, 7.75cm)   do tamanho das fontes
% 3 linhas: (4cm, 7.15cm)   e o espaçamento delas
% 4 linhas: (4cm, 6.55cm)
% 5+ linhas: diminuir tamanho da fonte pra caber em 4

\begin{textblock*}{16cm}(4cm, 8.38cm) %reposicione aqui

    %\fontsize{tamanho}{espaçamento de linha}
    {\fontsize{38}{22}\selectfont Título show \par}
    %aplique \\ para saltar linhas pra rearranjar as palavras do título e ficarem melhor distribuídas
    
    %\par é para delimitar o parágrafo e poder aplicar efeitos de espaçamento entre linhas
\end{textblock*}
\end{flushright}
\end{minipage}

\vspace*{10cm}
    %{\fontsize{38}{150}\selectfont Análise do consumo médio    anual de obras de pavimentação no Distrito\par}

\begin{flushright}
\begin{minipage}{6cm} 
 \parbox[t]{8cm}{\textbf{Consultores Responsáveis:}\\ 
Renan Menezes \\
} \\ \\
\parbox[t]{5cm}{\textbf{Requerente:}\\ 
Projeto fantasma\\
}
\end{minipage}
\end{flushright}
\vspace{2cm}

\includegraphics[scale=0.48]{estat.png}

\vfill
\end{titlepage}

\tableofcontents
\thispagestyle{empty}
\newpage

\section{Pode colocar as sections normalmente a partir de agora}
\AddToShipoutPictureBG{\includegraphics[width=\paperwidth,height=\paperheight]{pagina-comum.png}}

A partir daqui é tudo normal, as mudanças são detalhes apenas!

A estrutura de relatório segue a mesma, as mudanças foram apenas a arte da capa e das páginas, a forma de apresentarmos alguns quadros (teste de hipóteses) e a forma que fazemos a conclusão.

\section{A mudança mais critícia era a da Capa}

Basta dar um crtl+c e um crtl+v em tudo entre a linha 1 e 122 (parte comentada "pacote + capa"), depois só mudar o título e as outras infos da capa como consultor e referente!

Cuidado caso tenha adicionado um pacote diferente pra não esquecer de adicionar ele depois de copiar, e não se preocupe em entender tudo o que foi feita nessa capa, funções e argumentos, etc

Não esqueça de baixar o capa.png, pagina-comum.png e o estat.png e upar no seu projeto fantasma do overleaf, é necessário pra mudar a capa e a arte das páginas restantes.

\section{Mudanças nos quadros de testes de hipóteses}

\begin{quadro}[H]
\centering
\caption{Nova forma de apresentar os resultados do teste, com apenas os p-valores}
\label{R-Q-Teste-1}
\vspace{0.1cm}
\resizebox{\textwidth}{!}{
\begin{tabular}{|c|c|c|c|c|}
  \hline
 \textbf{Variável} & \textbf{Teste de Normalidade} & \textbf{Decisão do Teste} & \textbf{Teste Exalta Samba} & \textbf{Decisão do Teste} \\ 
  \hline 
  Volume & (P-valor) & Não rejeita $H_0$ & (P-valor) & Rejeita $H_0$\\
  \hline
\end{tabular}}
\end{quadro}

Mudanças na forma de apresentar os resultados do teste, nos apresentamos somente as hipóteses do teste principal, o teste pra pressupostos apenas é comentado nos parágrafos (não apresentamos as hipóteses pra os pressupostos como fazemos com o teste normal).

E aí, agora colocamos junto em um quadro a variável testada, os pressupostos testados ou o teste desejado, e a decisão com base no p-valor.

Como caption, geralmente é colocado: "P-valores das hipóteses testadas sobre tantantantan"

\section{Mudanças na forma de concluir}

Antes a forma de concluir o relatório, era em tópicos, agora fazemos a conclusão em parágrafos mesmo. A ideia da conclusão é que ela apresente de forma sucinta e objetiva o objetivo do cliente com a consultoria - imaginamos que o cliente queira saber qual o impacto de um tratamento nas lesões dos pacientes, e a ideia seria fazer uma conclusão que apresenta exatamente seus objetivos: parágrafos explicando em quais tratamentos a intervenção foi eficaz ou se piorou a dor dos pacientes.

Aqui vai um exemplo: 

https://www.ime.usp.br/~jmsinger/MAE0217/RelatorioCEA08P27.pdf

\end{document}
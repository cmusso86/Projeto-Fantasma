\documentclass[a4paper, 12pt]{article} %tipo de documento e opções gerais
\usepackage[utf8]{inputenc} %codificação
\usepackage[T1]{fontenc} %codificação
\usepackage[portuguese]{babel}  %idioma
\usepackage{amsmath,amssymb,amsfonts,amsthm} %padrões em formato matemático
\usepackage[a4paper,top=3cm,bottom=2cm,left=3cm, right=2cm]{geometry} %margens
\usepackage{indentfirst} %primeiro parágrafo com margem
\usepackage{float} %fixar imagens e tabelas
\usepackage{multicol} %várias colunas
\usepackage{multirow} %várias linhas
\usepackage{graphicx} %colocar imagens 
\usepackage{anyfontsize} %qualquer tamanho de letra
\usepackage{setspace} %espaçamento
\usepackage[titles]{tocloft} %padrões de sumário
\usepackage{fontspec} %outros tipos de fonte
\usepackage{fancyhdr} %padronizar o formato do Header
\usepackage[resetlabels,labeled]{multibib} %bibliografia
\usepackage{newfloat} %alterar nome do quadro
\usepackage{tabls} %pacote para espaçamento de linhas
%\usepackage{titlesec}
\usepackage{makecell}
\usepackage{shadowtext}
\usepackage{eso-pic,graphicx}
\usepackage{tikz} %trabalhar com imagens e add background
\usepackage[absolute,overlay]{textpos}
\usepackage{calc} %para funçoes como \widthof
\usepackage{hyperref}
\usepackage{afterpage} % contracapa

\newcommand\blankpage{% código para a contracapa
    \null
    \thispagestyle{empty}%
    \addtocounter{page}{-1}%
    \newpage}

%definindo novo estilo de pagina
\makeatletter
\def\ps@Padrao{
    %número da página com cor clara e no canto
    \def\@oddfoot{\textcolor{white}{\null\hfill\thepage}}%
    \def\@evenfoot{\thepage}%
    %definindo cabeçado para o canto
    \def\@evenhead{\null\hfil\slshape\leftmark }%
    \def\@oddhead{{\slshape\rightmark\hfill \includegraphics[scale=0.2]{estat.png}}}} %cabeçalho
\makeatother

\pagestyle{Padrao}

\setmainfont{Arial} %fonte arial
\setstretch{1.5} %espaçamento
\setlength\tablinesep{5pt} %espaço entre as células da tabela

% ALTERANDO O TÍTULO DAS TABELAS E FIGURAS
\addto\captionsenglish{%
  \renewcommand\tablename{Tabela}
  \renewcommand\figurename{Figura}
}
\DeclareFloatingEnvironment[listname=loq, listname={Lista de Quadros}]{quadro}

% ALTERANDO O SUMÁRIO
\makeatletter
\renewcommand\tableofcontents{
  \null\hfill\textbf{\Large\contentsname}\hfill\null\par
  \@mkboth{\MakeUppercase\contentsname}{\MakeUppercase\contentsname}%
  \@starttoc{toc}}
\makeatother	
\addto\captionsenglish{
  \renewcommand{\contentsname}{Sumário}
  }

\clearpage

\begin{document}
\begin{titlepage}

\center
\tikz[remember picture,overlay] \node[opacity=1,inner sep=0pt] at (current page.center){\includegraphics[width=\paperwidth,height=\paperheight]{capa.png}};

\begin{minipage}{16cm}
\begin{flushright}
% posições do título (substituir o segundo parâmetro do begin{textblock}
% 1 linha:  (4cm, 8.38cm)   nao esquecer de cuidar
% 2 linhas: (4cm, 7.75cm)   do tamanho das fontes
% 3 linhas: (4cm, 7.15cm)   e o espaçamento delas
% 4 linhas: (4cm, 6.55cm)
% 5+ linhas: diminuir tamanho da fonte pra caber em 4

\begin{textblock*}{16cm}(4cm, 8.38cm) %reposicione aqui

    %\fontsize{tamanho}{espaçamento de linha}
    {\fontsize{38}{22}\selectfont Avaliação de Filmes em Plataformas de Streaming \par}
    %aplique \\ para saltar linhas pra rearranjar as palavras do título e ficarem melhor distribuídas
    
    %\par é para delimitar o parágrafo e poder aplicar efeitos de espaçamento entre linhas
\end{textblock*}
\end{flushright}
\end{minipage}

\vspace*{10cm}
    %{\fontsize{38}{150}\selectfont Análise do consumo médio    anual de obras de pavimentação no Distrito\par}

\begin{flushright}
\begin{minipage}{6cm} 
 \parbox[t]{8cm}{\textbf{Consultora Responsável:}\\ 
Carolina Musso \\
} \\ \\
\parbox[t]{5cm}{\textbf{Requerente:}\\ 
Floriano, Diretor de Expansão da VocêTudo\\
}
\end{minipage}
\end{flushright}
\vspace{2cm}

\includegraphics[scale=0.48]{estat.png}

\vfill
\end{titlepage}

\tableofcontents
\thispagestyle{empty}
\newpage

\section{Introdução}
\AddToShipoutPictureBG{\includegraphics[width=\paperwidth,height=\paperheight]{pagina-comum.png}}

O seguinte projeto tem o objetivo de avaliar atributos de filmes que estão disponíveis em plataformas de Streaming. Tais análises visam servir de insumo para tomadas de decisão da empresa VocêTubo, que está planejando oferecer como serviço uma nova Plataforma de streaming.  

Especificamente, pretende-se avaliar:\\
- A quantidade de lançamentos ao longo dos anos;\\
- Se há alguma relação entre a duração dos filmes e o seu ano de lançamento;\\
- Se a nota do Rotten Tomatoes do filme é influenciada pelo tempo de duração dele;\\
- A distribuição de classificação indicativa do filme por plataforma;
Uma comparação entre o IMDb das plataformas.

Para tal, será realizado uma análise descritiva das variáveis pertinentes presentes no banco, bem como teste de hipótese para avaliar possíveis correlações entre variáveis contínuas, associações entre a variáveis categóricas, testes para comparação de médias entre grupos, e regressões com modelos lineares generalizados, de acordo com os objetivos.

A Base de Dados utilizada foi oferecida pelo cliente e possui dados de 16744 filmes, para os quais se tem as seguintes informações: Título, Ano de Lençamento, Classificação Indicativa do FIlme, Notas no IMDb e Rotten Tomatoes, Quais plataformas estão disponíveis (Netflix, PrimeVideo, Hulu, Disney+, Nome do(s) Diretor(es), Gênero do Filme (Ação, Drama, Terror, etc), País, Língua e Tempo de duração em minutos. 

Para as análises foi utilizado o softwarer R versão 4.0.3 (10/10/2020) e os pacotes do tidyverse bem como os pacotes ggpubr, papeR, moments, 
e wordcloud.

\section{Metodologia}

\subsection{Medidas}
Foram utilizadas as seguintes medidas de posição e dispersão:\\
\textbf{Média:}\\
\textbf{Desvio Padrão}\\
Quartis e Mediana:\\
Min Max:\\
Curtose:\\
Assimetria:\\

\subsection{Gráficos Utilizados}
Boxplot:\\
Gráfico de Dispersão:\\
QQ-Plot:\\
Curva de Densidade:\\
Gráfico de Colunas ou Barras:\\
Gráfico de Setores:\\

\subsection{Teses de Hipótese}

Chiquadrado:\\
Kruskal-Wallis:\\
Shapiro-Wilk:\\
p-valor\\
GLM:\\
Nuvem de Palavras:\\

\section{Análises}

\begin{quadro}[H]
\centering
\caption{Nova forma de apresentar os resultados do teste, com apenas os p-valores}
\label{R-Q-Teste-1}
\vspace{0.1cm}
\resizebox{\textwidth}{!}{
\begin{tabular}{|c|c|c|c|c|}
  \hline
 \textbf{Variável} & \textbf{Teste de Normalidade} & \textbf{Decisão do Teste} & \textbf{Teste Exalta Samba} & \textbf{Decisão do Teste} \\ 
  \hline 
  Volume & (P-valor) & Não rejeita $H_0$ & (P-valor) & Rejeita $H_0$\\
  \hline
\end{tabular}}
\end{quadro}

Mudanças na forma de apresentar os resultados do teste, nos apresentamos somente as hipóteses do teste principal, o teste pra pressupostos apenas é comentado nos parágrafos (não apresentamos as hipóteses pra os pressupostos como fazemos com o teste normal).

E aí, agora colocamos junto em um quadro a variável testada, os pressupostos testados ou o teste desejado, e a decisão com base no p-valor.

Como caption, geralmente é colocado: "P-valores das hipóteses testadas sobre tantantantan"

\section{Conclusões}

Antes a forma de concluir o relatório, era em tópicos, agora fazemos a conclusão em parágrafos mesmo. A ideia da conclusão é que ela apresente de forma sucinta e objetiva o objetivo do cliente com a consultoria - imaginamos que o cliente queira saber qual o impacto de um tratamento nas lesões dos pacientes, e a ideia seria fazer uma conclusão que apresenta exatamente seus objetivos: parágrafos explicando em quais tratamentos a intervenção foi eficaz ou se piorou a dor hhahahahahaha.

Aqui vai um exemplo: 

https://www.ime.usp.br/~jmsinger/MAE0217/RelatorioCEA08P27.pdf

\end{document}